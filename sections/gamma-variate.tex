% gamma-variate.tex
%
% Drafted by Juntang on August 20, 2024
% Adjusted by Juntang on August 22, 2024

\begin{frame}
  \frametitle{Gamma Variate Function}
    \cite{thompsonIndicatorTransitTime1964}
    \begin{columns}
    %% double-column layout 
    \begin{column}{0.45\textwidth}                 %% Left Column 
    {\small
    \begin{itemize}
        \item GVF \& Adj. Sheppard's model:
    \end{itemize}
    \begin{equation}
    C(t) = k(t - AT)^{\alpha} e^{-\frac{(t - AT)}{\sigma}}
    \end{equation}
    \begin{equation}
    C(t) = \frac{A (t - AT)^{\alpha}}{\Gamma(1 + \alpha) \sigma^{1+\alpha}} e^{-\frac{(t - AT)}{\sigma}}
    \end{equation}

    $\begin{aligned}
        t &= \text{time after injection} \\
        C(t) &= \text{indicator concentration at time, } t \\
        k &= \text{constant scale factor} \\
        AT &= \text{appearance time} \\
        \alpha, \sigma &= \text{arbitrary parameters, } 1/\sigma = Q/V \\
        A &= \text{total area under the curve, } I/Q
    \end{aligned}$
    }
    \end{column}
    
    \hspace*{4em}                                                          %%  make space between columns 
    
    \begin{column}{0.42\textwidth}                                %% Right Column 
        Indicator transit time has been shown to exhibit the mathematical properties of a general class of random variables, known as "gamma variates." Curve-fitting techniques were employed to show that the arterial indicator curves are equivalent to frequency distribution functions for this class of variables.
    \end{column}
    %
    \end{columns}   %% End of multiple columns 
    % 

\end{frame}

%%% Local Variables:
%%% mode: latex
%%% TeX-master: "../topic-slide-main"
%%% End: